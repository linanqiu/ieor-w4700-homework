\documentclass[11pt]{scrartcl}
\usepackage{dominatrix}
\usepackage{solarized-light}
\lstset{
language=R,
basicstyle=\ttfamily\footnotesize
}
\renewcommand\thesection{Problem \arabic{section}}
\renewcommand\thesubsection{\thesection (\alph{subsection})}
\renewcommand\thesubsubsection{(\roman{subsubsection})}

\newcommand{\portfolio}[2]{\ensuremath{\dfrac{K_3-K_2}{K_3-K_1}#1 + \dfrac{K_2-K_1}{K_3-K_1}#2}}
\newcommand{\leftportfolio}[1]{\ensuremath{\dfrac{K_3-K_2}{K_3-K_1}#1}}

\newcommand{\defrac}[2]{\ensuremath{\frac{\delta #1}{\delta #2}}}
\newcommand{\dedefrac}[2]{\ensuremath{\frac{\delta^2 #1}{\delta #2^2}}}
\newcommand{\dededefrac}[3]{\ensuremath{\frac{\delta^2 #1}{\delta #2 \delta #3}}}
\renewcommand{\sf}{\ensuremath{Se^{(r-q)(T-t)}}}

\newcommand{\epower}[1]{\ensuremath{e^{\left(#1\right)}}}
\newcommand{\epowernb}[1]{\ensuremath{e^{#1}}}
\newcommand{\expectation}[1]{\ensuremath{\mathrm{E}{\left(#1\right)}}}

\title{Homework 10}
\subject{Intro to Financial Engineering IEOR W4700}
\author{Linan Qiu\\\texttt{lq2137}}
\begin{document}
\maketitle

\section{}

\[\epowernb{-y_2(T_2-t)} = \epowernb{-y_1(T_1-t)}\epowernb{-y_{1,2}(T_2-T_1)}\]

Then,

\[-y_2(T_2 -t) = -y_1(T_1-t) -y_{1,2}(T_2-T_1)\]

Or setting the subject to be the forward rate $y_{1,2}$,

\[y_{1,2} = \frac{-y_1(T_1-t) + y_2(T_2-t)}{T_2-T_1}\]

This function in \texttt{R}:

\begin{lstlisting}
forward_rate = function(T1, T2, t, y1, y2) {
  return ((-y1*(T1 - t) + y2*(T2-t))/(T2-T1))
}
\end{lstlisting}

Then, we recognize that for the problem, $t=0$ for all calculations, while $T_2$ and $T_1$ always have a difference of $\frac{1}{4}$ since we are advancing in quarters. Then we can do a shortcut as such:

\begin{lstlisting}
maturities = c(3, 6, 9, 12, 15, 18)
maturities = maturities / 12
spot_rates = c(7, 7.2, 7.3, 7.5, 7.7, 7.9)
spot_rates = spot_rates / 100

quarterly_forward_rate = function(period) {
  T2 = maturities[period]
  T1 = maturities[period - 1]
  
  y2 = spot_rates[period]
  y1 = spot_rates[period - 1]
  
  t = 0
  
  return(forward_rate(T1, T2, t, y1, y2))
}

forward_rates = sapply(c(2:6), quarterly_forward_rate)
\end{lstlisting}

To get the forward rates, we do

\begin{lstlisting}
> forward_rates
[1] 0.074 0.075 0.081 0.085 0.089
\end{lstlisting}

\section{}

Value of \$1 FRA can be replicated by longing $Z(T_1)$ and shorting $(1+K(T_2 - T_1))Z(T_2)$. The value of the FRA in the question is then

\[N\left[Z(T_1) -  (1+K(T_2 - T_1))Z(T_2)\right]\]

Now, using the spots given,

\[Z(T_1) = Z(12/12) = \epowernb{-0.085*1}\]

\[Z(T_2) = Z(15/12) = \epowernb{-0.086*1.25}\]

\[K = 0.095\]

Then, the value of the FRA is

\[2000000 * \left[\epowernb{-0.085} - (1+0.095(3/12))\epowernb{-0.086} \right] = -1787.111\]

\textbf{The FRA is worth} \$-1787.111.

To verify this answer, we can use the forward rates.

Using the same method as in the previous section, we find the quarterly forward rates as follows:

\begin{lstlisting}
> spot_rates = c(8, 8.2, 8.4, 8.5, 8.6, 8.7)
> forward_rates = sapply(c(2:6), quarterly_forward_rate)
> forward_rates
[1] 0.084 0.088 0.088 0.090 0.092
\end{lstlisting}

Then the annual forward rate for one quarter starting in 1 year is 0.090.

We can find the expiration value of the fixed leg: $0.095/4$. We can also find the expiration value of the floating leg: $\epower{0.09*0.25} - 1$. Then, by finding the present value of that difference and multiplying it by the notional, we arrive at the same answer.

\begin{lstlisting}
> fixedleg = 0.095/4
> floatleg = exp(0.09*0.25) - 1
> difference = floatleg - fixedleg
> difference_pv = difference * exp(-0.086*5/4)
> fra = difference_pv*2000000
> fra
[1] -1787.111
\end{lstlisting}

\section{}

\begin{lstlisting}
time_month = c(0, 6, 12, 18, 24, 30)
time = time_month / 12
time_payment_month = time_month + 6
time_payment = time_payment_month / 12

six_m_libor = c(5, 5.8, 5.3, 5.5, 5.6, 6.1)
six_m_libor = six_m_libor / 100
fixed_rate = 0.1
fixed_rate = c(rep(fixed_rate, 6))

fixed_payments = (fixed_rate / 2) * 10000000
floating_payments = (six_m_libor/ 2) * 10000000
net_payments = fixed_payments - floating_payments

table = data.frame(time_payment_month, fixed_rate , six_m_libor, fixed_payments, floating_payments, net_payments)
colnames(table) = c('Time', 'Fixed Rate', '6 Month Libor Rate', 'Fixed Payments', 'Floating Payments', 'Net Payments')
library(xtable)
xtable(table)
\end{lstlisting}

Table generated is reproduced on the next page.

\begin{landscape}

% latex table generated in R 3.1.2 by xtable 1.7-4 package
% Sun Dec  6 18:22:58 2015
\begin{table}[H]
\centering
\begin{tabu}{r|r|c|c|c|c|c}
\toprule
 & Time & Fixed Rate & 6 Month Libor Rate & Fixed Payments & Floating Payments & Net Payments \\
\midrule
1 & 6 & 0.10 & 0.05 & 500000.00 & 250000.00 & 250000.00 \\ 
  2 & 12 & 0.10 & 0.06 & 500000.00 & 290000.00 & 210000.00 \\ 
  3 & 18 & 0.10 & 0.05 & 500000.00 & 265000.00 & 235000.00 \\ 
  4 & 24 & 0.10 & 0.06 & 500000.00 & 275000.00 & 225000.00 \\ 
  5 & 30 & 0.10 & 0.06 & 500000.00 & 280000.00 & 220000.00 \\ 
  6 & 36 & 0.10 & 0.06 & 500000.00 & 305000.00 & 195000.00 \\ 
\bottomrule
\end{tabu}
\caption{Payment stream of IRS}
\end{table}

\end{landscape}

\section{}

\subsection{}

Current value of the IRS is the PV of the net payments.

\begin{lstlisting}
> value = sum(net_payments * discount_factors)
> value
[1] 1225252
\end{lstlisting}

\subsection{}

Current forward swap rate is the fixed rate that makes the swap value zero ie. the PV of the fixed leg equal to the PV of the floating leg.

\begin{lstlisting}
discount_factors = c(0.9778, 0.9541, 0.9291, 0.9048, 0.8781, 0.8479)

gradient_descent = function(k) {
  fixed_rate = c(rep(k, 6))
  fixed_payments = (fixed_rate / 2)
  floating_payments = (six_m_libor/2)
  pv_fixed = fixed_payments * discount_factors
  pv_float = floating_payments * discount_factors
  return(sum(pv_fixed) - sum(pv_float))
}
\end{lstlisting}

Using this function, we can calculate the fixed rate that makes this function 0.

\begin{lstlisting}
> uniroot(gradient_descent, c(0, 0.1))
$root
[1] 0.05537886

$f.root
[1] 0

$iter
[1] 1

$init.it
[1] NA

$estim.prec
[1] 0.05537886
\end{lstlisting}

$K = 0.05537886$

The closed form solution for this can be found by making $K$ the subject of this equation, where $V_\mathrm{swap}$ is the value of the swap found earlier by using $K=0.1$:

\[V_\mathrm{swap} = 1 - \left(Z(t, T_6) + \frac{K}{2}\sum_{i=1}^nZ(t, T_i) \right)\]

However, it was just easier to calculate it directly (though computationally inefficient).

\end{document}