\documentclass[11pt]{scrartcl}
\usepackage{dominatrix}
\usepackage{solarized-light}
\lstset{
language=R
}
\renewcommand\thesection{Problem \arabic{section}}
\renewcommand\thesubsection{(\alph{subsection})}
\renewcommand\thesubsubsection{(\roman{subsubsection})}
\title{Homework 2}
\subject{Intro to Financial Engineering IEOR W4700}
\author{Linan Qiu\\\texttt{lq2137}}
\begin{document}
\maketitle

\section{}

We can think of the house as a series of yearly payments $x$ for 20 years whose present value equates $20000$. Then,

\[20000 = \frac{x}{0.05}\left(1 - \frac{1}{(1+0.05)^{20}}\right)\]

Solving for $x$, $x = 1604.852$

\begin{lstlisting}
> f = function(x){(x/0.05) * (1-1/((1+0.05)^20)) - 20000}
> uniroot(f, c(-99999999, 99999999))
$root
[1] 1604.852

$f.root
[1] 4.366302e-08

$iter
[1] 2

$init.it
[1] NA

$estim.prec
[1] 6.103516e-05
\end{lstlisting}

Now, we know the yearly ``coupon'' of the roof. Since our current roof only has 5 years left, then the value of the house is the payment for 5 years.

\[PV_R = \sum_{i=1}^5 \frac{x}{(1+0.05)^i} = \frac{x}{0.05}\left(1 - \frac{1}{(1+0.05)^5} \right) = 6948.168\]

The value of the existing roof if $6948.168$

\section{}

\subsection{}

The stream of cashflow is

\[PV = -3x + \frac{5}{1+r} + \frac{x}{(1+r)^2}\]

For $PV > 0$, 

\begin{align*}
-3x + \frac{5}{1+r} + \frac{x}{(1+r)^2} &> 0 \\
-3x(1+r)^2 + 5(1+r) + x &> 0 \\
-3x(1+r)^2 + x &> -5(1+r) \\
x(-3(1+r)^2 + 1) &> -5(1+r) \\
x &> \frac{-5(1+r)}{-3(1+r)^2 + 1} \\
x &> \frac{5(1+r)}{3(1+r)^2 - 1}
\end{align*}

\subsection{}

IRR is the $r$ such that $PV = 0$ or

\[-3x + \frac{5}{1+r} + \frac{x}{(1+r)^2} = 0\]

Then, let $c = \frac{1}{1+r}$. 

\[-3x + 5c + xc^2\]

We need

\begin{align*}
r &> 0 \\
1 + r &> 1 \\
\frac{1}{1+r} &> 1 \\
c &> 1
\end{align*}

Solving quadratically,

\[c = \frac{-5 \pm\sqrt{5^2 - 4(-3x)(x)}}{2x} = \frac{-5 \pm\sqrt{25 + 12x^2}}{2x}\]

Discard the strictly negative root since it won't fulfill our condition of $c > 1$. Using the possibly positive root, we need $c>1$ or

\[\frac{-5 + \sqrt{25 + 12x^2}}{2x} > 1\]

$x = 0$ or $x = 2.5$

\begin{lstlisting}
> install.packages("rootSolve")
> library(rootSolve)
> g = function(x) {(-5+sqrt(25 + 12*x^2)) / (2*x) - 1}
> uniroot.all(g, c(-9, 9))
[1] 2.5
\end{lstlisting}

$x > 2.5$ guarantees a strictly positive IRR.

\section{}

This can be treated as monthly payments with interest rate $0.01$ since rent is paid monthly. Assume monthly compounding.

Compare PV:

\begin{itemize}
\item \textbf{Stay} $PV = \frac{1000}{0.01} \left(1 - \frac{1}{(1+0.01)^6}\right) = 5795.476$
\item \textbf{Switch} $PV = 1000 + \frac{900}{0.01} \left(1 - \frac{1}{(1+0.01)^6} \right) = 6215.929$
\end{itemize}

\textbf{Within 6 months, the couple should stay} since staying costs them less in PV terms.

For 1 year, compare PV again:

\begin{itemize}
\item \textbf{Stay} $PV = \frac{1000}{0.01} \left(1 - \frac{1}{(1+0.01)^12}\right) = 11255.08$
\item \textbf{Switch} $PV = 1000 + \frac{900}{0.01} \left(1 - \frac{1}{(1+0.01)^6} \right) = 11129.57$
\end{itemize}

\textbf{For 1 year, the couple should switch} since switching costs them less in PV terms.

\section{}

$x_K$ is the number of put options purchased whose strike price is $K$. Let $x_L$ is the number of put options purchased whose strike price is $L$. To arb this, find $x_K$ and $x_L$ such that

\begin{align*}
-P(t, T, K)x_K - P(t, T, L)x_L &\geq 0 \\
\max\left[0, (K-S(T))\right]x_K + \max\left[0, (L-S(T))\right]x_L &\geq 0
\end{align*}

Both inequalities need to hold true (since one represents time $t$ and the other time of expiration $T$). Furthermore, one of the inequalities need to be strict (for us to be able to profit not just break even).

The $\max$ notation is really annoying, so let's get rid of it by specifying ranges of $S(T)$.

\begin{itemize}
\item When $L < S(T)$

$x_K$ and $x_L$ are 0, holding both inequalities true.

\item When $K < S(T) < L$

The two inequalities become:

\begin{align*}
-P(t, T, K)x_K - P(t, T, L)x_L &\geq 0 \\
(L-S(T))x_L &\geq 0
\end{align*}

Then, $x_L \geq 0$ from the second inequality.

\item When $S(T) < K$

The two inequalities become:

\begin{align*}
-P(t, T, K)x_K - P(t, T, L)x_L &\geq 0 \\
(K-S(T))x_K + (L-S(T))x_L &\geq 0
\end{align*}

From the first inequality, $x_K \leq \frac{-P(t, T, L)x_L}{P(t, T, K)}$

From the second inequality, $x_K \geq \frac{-(L-S(T))x_L}{(K-S(T))}$. Then, we require that

\[\frac{-(L-S(T))x_L}{(K-S(T))} \leq x_K \leq \frac{-P(t, T, L)x_L}{P(t, T, K)}\]

Now if $\frac{P(t, T, L)}{P(t, T, K)} > \frac{L}{K}$, then $\frac{-P(t, T, L)}{P(t, T, K)} < \frac{-L}{K}$. Then, we can find a small $S(T)$ such that $\frac{-P(t, T, L)}{P(t, T, K)} < \frac{-(L-S(T))}{(K-S(T))}$. Then there would be no solution for $x_K$ from the above two inequalities.
\end{itemize}

Thus there is no arbitrage strategy.

\section{}

\subsection{}

\[PV = \frac{1}{(1+r)} + \frac{2}{(1+r)^2} + ... + \frac{N}{(1+r)^N}\]

Denote present value with $n$ periods as $P(n)$. Then, $P(0) = 0$.

\begin{align*}
P(N) &= \frac{1}{(1+r)} + \frac{2}{(1+r)^2} + ... + \frac{N}{(1+r)^N} \\
&= \frac{1}{(1+r)} + \frac{1}{(1+r)^2} + ... + \frac{1}{(1+r)^N} + \frac{N-1}{(1+r)^N} \\
&= \frac{1}{r}\left(1-\frac{1}{(1+r)^N}\right) + \frac{P(N-1)}{(1+r)} \\
&= \frac{1}{r}\left(1-\frac{1}{(1+r)^N}\right) + \frac{P(N)}{(1+r)} - \frac{N}{(1+r)^{N+1}} \\
P(N) \left(1 - \frac{1}{1+r}\right) &= \frac{1}{r}\left(1-\frac{1}{(1+r)^N}\right) - \frac{N}{(1+r)^{N+1}} \\
P(N) &= \frac{\frac{1}{r}\left(1-\frac{1}{(1+r)^N}\right) - \frac{N}{(1+r)^{N+1}}}{\left(1 - \frac{1}{1+r}\right)}
\end{align*}

Please don't make me simplify this? It's midnight, and also there's no need to because there's \texttt{r}?

\subsection{}

Now,

\begin{align*}
\lim_{N \to \infty} \frac{N}{(1+r)^{N+1}} &= \frac{\lim_{N \to \infty} N}{\lim_{N \to \infty} (1+r)^{N+1}} \\
&= \frac{\lim_{N\to\infty}1}{\lim_{N\to\infty}(1+r)^{N+1} \log{(1+r)}} \\
&= 0
\end{align*}

\begin{align*}
\lim_{N \to \infty} P(N) &= \lim_{N \to \infty} \frac{\frac{1}{r}\left(1-\frac{1}{(1+r)^N}\right) - \frac{N}{(1+r)^{N+1}}}{\left(1 - \frac{1}{1+r}\right)} \\
&= \frac{\frac{1}{r}}{1-\frac{1}{1+r}} \\
&= \frac{1+r}{r^2}
\end{align*}

\section{}

\begin{itemize}
\item To calculate 1 year zero rate, $95 = \frac{100}{(1+r_1)}$, then $r_1 = 0.05262625$
\item To calculate 2 year zero rate, $90 = \frac{100}{(1+r_2)^2}$, then $r_2 = 0.05409221$
\end{itemize}

We now have a yield curve for 2 years.

Cashflow of the two-year bond can be stripped:

\[PV = \frac{10}{1+r_1} + \frac{100 + 10}{(1+r_2)^2} = 108.5\]

The fair value is $108.5$

\section{}

Project 1:

\[0 = -A_1 + \frac{B_1}{1+r_1} + \frac{B_1}{(1+r_1)^2} + ... + \frac{B_1}{(1+r_1)^n} = -A_1 + \frac{B_1}{r_1} \left(1-\frac{1}{(1+r_1)^n}\right)\]

implies that

\begin{align*}
0 &= -A_1 + \frac{B_1}{r_1} \left(1-\frac{1}{(1+r_1)^n} \right)\\
\frac{B_1}{A_1} &= \frac{r_1}{\left(1-\frac{1}{(1+r_1)^n} \right)} \\
&= \frac{r_1 (1+r_1)}{(1+r_1)^n} \\
&= \frac{r_1}{(1+r_1)^{n-1}}
\end{align*}

Project 2:

\[0 = -A_2 + \frac{B_2}{1+r_2} + \frac{B_2}{(1+r_2)^2} + ... + \frac{B_2}{(1+r_2)^n} = -A_2 + \frac{B_2}{r_2} \left(1-\frac{1}{(1+r_2)^n}\right)\]

Since $\frac{B_1}{A_1} > \frac{B_2}{A_2}$, then

\begin{align*}
\frac{r_1}{(1+r_1)^{n-1}} > \frac{r_2}{(1+r_2)^{n-1}}\\
r_1 > r_2
\end{align*}

Hence project 1 will have a higher IRR than project 2.

\section{}

Monthly payment $A$ is

\[203150 = \frac{A}{\left( \frac{0.08083}{12} \right)} \left(1 - \frac{1}{\left(1+\frac{0.08083}{12}\right)^{30*12}} \right)\]

Solving for $A$, $A = 1504.76$.

\begin{lstlisting}
> f = function(a) {(a/(0.08083/12)) * (1-1/(1+0.08083/12)^360) - 203150}
> uniroot(f, c(-99999999999, 99999999999))
$root
[1] 1502.414

$f.root
[1] -7.888876e-05

$iter
[1] 2

$init.it
[1] NA

$estim.prec
[1] 6.103516e-05
\end{lstlisting}

The principal that corresponds with $A$ and rate of $7.875\%$ is

\[P = \frac{A}{\left( \frac{0.07875}{12} \right)} \left(1 - \frac{1}{\left(1+\frac{0.07875}{12}\right)^{30*12}} \right)\]

Solving for $P$ using $A = 1502.414$, $P = 207209.7$.

Then, total fee is $207209.7 - 203150 = 4059.7$.

\section{}

Coupon $C = 5\%$. Assume, without loss of generality, that the bond face value is $1$. Let yield $r = 2x$.

After 5 years, the price of the bond should be 

\[P = \frac{0.05}{x} \left(1 - \frac{1}{(1 + x)^{30}} \right) + \frac{1}{(1+x)^{30}}\]

If the company exercises the call option, the company would have to pay $1.05$ per \$1 of bonds. That means $x$ would have to be such that $1.05 < P$ or

\[1.05 < \frac{0.05}{x} \left(1 - \frac{1}{(1 + x)^{30}} \right) + \frac{1}{(1+x)^{30}}\]

Solving for $x$, we get $x < 0.04686262$ or $r < 2x = 0.09372524$.

\begin{lstlisting}
> f = function(x) {(0.05/x)*(1-1/(1+x)^30) + 1/(1+x)^30 - 1.05}
> uniroot(f, c(0.000000001,9999999999))
$root
[1] 0.04686262

$f.root
[1] 3.102862e-06

$iter
[1] 41

$init.it
[1] NA

$estim.prec
[1] 0.0001155707

> x = uniroot(f, c(0.000000001,9999999999))
> x = x$root
> 2*x
[1] 0.09372524
\end{lstlisting}

Yield rates would have to be less than $9.372524\%$

\section{}

Ten year bonds have semi-annual coupon payments. Let price of bond be $P$.

\[P = \frac{4}{\frac{y}{2}} \left( 1 - \frac{1}{(1+\frac{y}{2})^{20}}\right) + \frac{100}{(1+\frac{y}{2})^{20}} = 87.53779\]

Substituting in $y = 0.1$, 

\[P = \frac{4}{0.05} \left( 1 - \frac{1}{1.05^{20}}\right) + \frac{100}{1.05^{20}} = 87.53779\]

Calculating numerical derivative at $y = 0.1$, $\frac{d P}{d y} = -570.2769$. Then, modified duration

\[D = - \frac{1}{P}\frac{dP}{dy} = - \frac{1}{87.53779}(-570.2769) = 6.514637\]

\begin{lstlisting}
> install.packages("numDeriv")
> library(numDeriv)
> f = function(y) {(4/(y/2)) * (1-1/(1+y/2)^20) + 100/(1+y/2)^20}
> z = grad(f, y, method="Richardson")
> z
[1] -570.2769
\end{lstlisting}

\end{document}
