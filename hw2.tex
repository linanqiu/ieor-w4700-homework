\documentclass[11pt]{scrartcl}
\usepackage{dominatrix}
\usepackage{solarized-light}
\lstset{
language=R
}
\renewcommand\thesection{Problem \arabic{section}}
\renewcommand\thesubsection{(\alph{subsection})}
\renewcommand\thesubsubsection{(\roman{subsubsection})}
\title{Homework 2}
\subject{Intro to Financial Engineering IEOR W4700}
\author{Linan Qiu\\\texttt{lq2137}}
\begin{document}
\maketitle

\section{}

We can think of the house as a series of yearly payments $x$ for 20 years whose present value equates $20000$. Then,

\[20000 = \frac{x}{0.05}\left(1 - \frac{1}{(1+0.05)^{20}}\right)\]

Solving for $x$, $x = 1604.852$

\begin{lstlisting}
> f = function(x){(x/0.05) * (1-1/((1+0.05)^20)) - 20000}
> uniroot(f, c(-99999999, 99999999))
$root
[1] 1604.852

$f.root
[1] 4.366302e-08

$iter
[1] 2

$init.it
[1] NA

$estim.prec
[1] 6.103516e-05
\end{lstlisting}

Now, we know the yearly ``coupon'' of the roof. Since our current roof only has 5 years left, then the value of the house is the payment for 5 years.

\[PV_R = \sum_{i=1}^5 \frac{x}{(1+0.05)^i} = \frac{x}{0.05}\left(1 - \frac{1}{(1+0.05)^5} \right) = 6948.168\]

The value of the existing roof if $6948.168$

\section{}

\subsection{}

The stream of cashflow is

\[PV = -3x + \frac{5}{1+r} + \frac{x}{(1+r)^2}\]

For $PV > 0$, 

\begin{align*}
-3x + \frac{5}{1+r} + \frac{x}{(1+r)^2} &> 0 \\
-3x(1+r)^2 + 5(1+r) + x &> 0 \\
-3x(1+r)^2 + x &> -5(1+r) \\
x(-3(1+r)^2 + 1) &> -5(1+r) \\
x &> \frac{-5(1+r)}{-3(1+r)^2 + 1} \\
x &> \frac{5(1+r)}{3(1+r)^2 - 1}
\end{align*}

\subsection{}

IRR is the $r$ such that $PV = 0$ or

\[-3x + \frac{5}{1+r} + \frac{x}{(1+r)^2} = 0\]

Then, let $c = \frac{1}{1+r}$. 

\[-3x + 5c + xc^2\]

We need

\begin{align*}
r &> 0 \\
1 + r &> 1 \\
\frac{1}{1+r} &> 1 \\
c &> 1
\end{align*}

Solving quadratically,

\[c = \frac{-5 \pm\sqrt{5^2 - 4(-3x)(x)}}{2x} = \frac{-5 \pm\sqrt{25 + 12x^2}}{2x}\]

Discard the strictly negative root since it won't fulfill our condition of $c > 1$. Using the possibly positive root, we need $c>1$ or

\[\frac{-5 + \sqrt{25 + 12x^2}}{2x} > 1\]

$x = 0$ or $x = 2.5$

\begin{lstlisting}
> install.packages("rootSolve")
> library(rootSolve)
> g = function(x) {(-5+sqrt(25 + 12*x^2)) / (2*x) - 1}
> uniroot.all(g, c(-9, 9))
[1] 2.5
\end{lstlisting}

$x > 2.5$ guarantees a strictly positive IRR.

\section{}

This can be treated as monthly payments with interest rate $0.01$ since rent is paid monthly. Assume monthly compounding.

Compare PV:

\begin{itemize}
\item \textbf{Stay} $PV = \frac{1000}{0.01} \left(1 - \frac{1}{(1+0.01)^6}\right) = 5795.476$
\item \textbf{Switch} $PV = 1000 + \frac{900}{0.01} \left(1 - \frac{1}{(1+0.01)^6} \right) = 6215.929$
\end{itemize}

\textbf{Within 6 months, the couple should stay} since staying costs them less in PV terms.

For 1 year, compare PV again:

\begin{itemize}
\item \textbf{Stay} $PV = \frac{1000}{0.01} \left(1 - \frac{1}{(1+0.01)^12}\right) = 11255.08$
\item \textbf{Switch} $PV = 1000 + \frac{900}{0.01} \left(1 - \frac{1}{(1+0.01)^6} \right) = 11129.57$
\end{itemize}

\textbf{For 1 year, the couple should switch} since switching costs them less in PV terms.

\section{}

\begin{itemize}
\item At expiration, $P(t, T, K)$ pays $\max{[0, K-S(T)]}$
\item At expiration, $P(t, T, L)$ pays $\max{[0, L-S(T)]}$
\end{itemize}

\section{}

\subsection{}

\[PV = \frac{1}{(1+r)} + \frac{2}{(1+r)^2} + ... + \frac{N}{(1+r)^N}\]

Denote present value with $n$ periods as $P(n)$. Then, $P(0) = 0$.

\begin{align*}
P(N) &= \frac{1}{(1+r)} + \frac{2}{(1+r)^2} + ... + \frac{N}{(1+r)^N} \\
&= \frac{1}{(1+r)} + \frac{1}{(1+r)^2} + ... + \frac{1}{(1+r)^N} + \frac{N-1}{(1+r)^N} \\
&= \frac{1}{r}\left(1-\frac{1}{(1+r)^N}\right) + \frac{P(N-1)}{(1+r)} \\
&= \frac{1}{r}\left(1-\frac{1}{(1+r)^N}\right) + \frac{P(N)}{(1+r)} - \frac{N}{(1+r)^{N+1}} \\
P(N) \left(1 - \frac{1}{1+r}\right) &= \frac{1}{r}\left(1-\frac{1}{(1+r)^N}\right) - \frac{N}{(1+r)^{N+1}} \\
P(N) &= \frac{\frac{1}{r}\left(1-\frac{1}{(1+r)^N}\right) - \frac{N}{(1+r)^{N+1}}}{\left(1 - \frac{1}{1+r}\right)}
\end{align*}

Please don't make me simplify this? It's midnight, and also there's no need to because there's \texttt{r}?

\subsection{}

Now,

\begin{align*}
\lim_{N \to \infty} \frac{N}{(1+r)^{N+1}} &= \frac{\lim_{N \to \infty} N}{\lim_{N \to \infty} (1+r)^{N+1}} \\
&= \frac{\lim_{N\to\infty}1}{\lim_{N\to\infty}(1+r)^{N+1} \log{(1+r)}} \\
&= 0
\end{align*}

\begin{align*}
\lim_{N \to \infty} P(N) &= \lim_{N \to \infty} \frac{\frac{1}{r}\left(1-\frac{1}{(1+r)^N}\right) - \frac{N}{(1+r)^{N+1}}}{\left(1 - \frac{1}{1+r}\right)} \\
&= \frac{\frac{1}{r}}{1-\frac{1}{1+r}} \\
&= \frac{1+r}{r^2}
\end{align*}

\section{}

\begin{itemize}
\item To calculate 1 year zero rate, $95 = \frac{100}{(1+r_1)}$, then $r_1 = 0.05262625$
\item To calculate 2 year zero rate, $90 = \frac{100}{(1+r_2)^2}$, then $r_2 = 0.05409221$
\end{itemize}

Cashflow of the two-year bond can be stripped:

\[PV = \frac{10}{1+r_1} + \frac{100 + 10}{1+r_2} = 113.8552\]

The fair value is $113.8552$

\section{}

Project 1:

\[0 = -A_1 + \frac{B_1}{1+r_1} + \frac{B_1}{(1+r_1)^2} + ... + \frac{B_1}{(1+r_1)^n} = -A_1 + \frac{B_1}{r_1} \left(1-\frac{1}{(1+r_1)^n}\right)\]

implies that

\begin{align*}
0 &= -A_1 + \frac{B_1}{r_1} \left(1-\frac{1}{(1+r_1)^n} \right)\\
\frac{B_1}{A_1} &= \frac{r_1}{\left(1-\frac{1}{(1+r_1)^n} \right)} \\
&= \frac{r_1 (1+r_1)}{(1+r_1)^n} \\
&= \frac{r_1}{(1+r_1)^{n-1}}
\end{align*}

Project 2:

\[0 = -A_2 + \frac{B_2}{1+r_2} + \frac{B_2}{(1+r_2)^2} + ... + \frac{B_2}{(1+r_2)^n} = -A_2 + \frac{B_2}{r_2} \left(1-\frac{1}{(1+r_2)^n}\right)\]

Since $\frac{B_1}{A_1} > \frac{B_2}{A_2}$, then

\begin{align*}
\frac{r_1}{(1+r_1)^{n-1}} > \frac{r_2}{(1+r_2)^{n-1}}\\
r_1 > r_2
\end{align*}

Hence project 1 will have a higher IRR than project 2.

\end{document}
