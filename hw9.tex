\documentclass[11pt]{scrartcl}
\usepackage{dominatrix}
\usepackage{solarized-light}
\lstset{
language=R,
basicstyle=\ttfamily\footnotesize
}
\renewcommand\thesection{Problem \arabic{section}}
\renewcommand\thesubsection{\thesection (\alph{subsection})}
\renewcommand\thesubsubsection{(\roman{subsubsection})}

\newcommand{\portfolio}[2]{\ensuremath{\dfrac{K_3-K_2}{K_3-K_1}#1 + \dfrac{K_2-K_1}{K_3-K_1}#2}}
\newcommand{\leftportfolio}[1]{\ensuremath{\dfrac{K_3-K_2}{K_3-K_1}#1}}

\newcommand{\defrac}[2]{\ensuremath{\frac{\delta #1}{\delta #2}}}
\newcommand{\dedefrac}[2]{\ensuremath{\frac{\delta^2 #1}{\delta #2^2}}}
\newcommand{\dededefrac}[3]{\ensuremath{\frac{\delta^2 #1}{\delta #2 \delta #3}}}
\renewcommand{\sf}{\ensuremath{Se^{(r-q)(T-t)}}}

\newcommand{\epower}[1]{\ensuremath{e^{\left(#1\right)}}}
\newcommand{\expectation}[1]{\ensuremath{\mathrm{E}{\left(#1\right)}}}

\title{Homework 9}
\subject{Intro to Financial Engineering IEOR W4700}
\author{Linan Qiu\\\texttt{lq2137}}
\begin{document}
\maketitle

\section{}

By put call parity,

\[C - P = S_0\epower{-qT} - K\epower{-rT}\]

Then, given the price of the call, $S_t$ and $K$ and $r$ and $T$,

\[P = C - S_0\epower{-qT} + K\epower{-rT} = 10 - 250\epower{-0.04*\frac{1}{r}} + 245\epower{-0.06*\frac{1}{4}} = 3.839967\]

\section{}

The adjusted stock price accounting for the dividend is

\[100 - 2\epower{-0.06*\frac{1}{6}} = 98.0199\]

Evolving the stock price using a binomial tree as follows:

\begin{lstlisting}
build_stock_tree = function(S, sigma, delta_t, N) {
  tree = matrix(0, nrow=N+1, ncol=N+1)
  
  u = exp(sigma*sqrt(delta_t))
  d = exp(-sigma*sqrt(delta_t))
  
  for (i in 1:(N+1)) {
    for (j in 1:i) {
      tree[i,j] = S * u^(j-1) * d^((i-1)-(j-1))
    }
  }
  return(tree)
}
\end{lstlisting}

Running the following code:

\begin{lstlisting}
> stock_tree = build_stock_tree(S=100 - 2*exp(-0.06/6), 0.3, 1/12, 4)
> stock_tree
         [,1]      [,2]     [,3]     [,4]     [,5]
[1,] 98.01990   0.00000   0.0000   0.0000   0.0000
[2,] 89.88832 106.88709   0.0000   0.0000   0.0000
[3,] 82.43132  98.01990 116.5564   0.0000   0.0000
[4,] 75.59294  89.88832 106.8871 127.1005   0.0000
[5,] 69.32186  82.43132  98.0199 116.5564 138.5984
\end{lstlisting}

This is the stock tree after adjusting for dividends.

Now we have to add the discounted dividend back to the tree. The first row gets $2\epower{-0.06*\frac{2}{12}}$ added, and the second row gets $2\epower{-0.06*\frac{1}{12}}$ back.

\begin{lstlisting}
> stock_tree[1,1] = stock_tree[1,1] + 2*exp(-0.06*2/12)
> stock_tree
          [,1]      [,2]     [,3]     [,4]     [,5]
[1,] 100.00000   0.00000   0.0000   0.0000   0.0000
[2,]  89.88832 106.88709   0.0000   0.0000   0.0000
[3,]  82.43132  98.01990 116.5564   0.0000   0.0000
[4,]  75.59294  89.88832 106.8871 127.1005   0.0000
[5,]  69.32186  82.43132  98.0199 116.5564 138.5984
> stock_tree[2,1] = stock_tree[2,1] + 2*exp(-0.06*1/12)
> stock_tree[2,2] = stock_tree[2,2] + 2*exp(-0.06*1/12)
> stock_tree
          [,1]      [,2]     [,3]     [,4]     [,5]
[1,] 100.00000   0.00000   0.0000   0.0000   0.0000
[2,]  91.87834 108.87712   0.0000   0.0000   0.0000
[3,]  82.43132  98.01990 116.5564   0.0000   0.0000
[4,]  75.59294  89.88832 106.8871 127.1005   0.0000
[5,]  69.32186  82.43132  98.0199 116.5564 138.5984
\end{lstlisting}

The resulting binary tree is shown directly above.

\section{}

Let the value of the binary option at time $0$ be $B_0$.

\[B_0 = \epower{-rT} \expectation{\theta(S_T - K)} = \epower{-rT} P(S_T \geq K)\]

Now the probability that $S_T \geq K$ can be found by looking at the distribution of $\log{S}$, since $P(S_T \geq K) = P(\log{S_T} \geq \log{K})$.

We know that $S_T$ is log-normally distributed, hence $\log{S_T}$ is normally distributed with the following statistics:

\[\expectation{\log{S_T}} = \log{S_0} + \left(\mu - \frac{1}{2} \sigma^2\right)T\]

\[\mathrm{Var}\left(\log{S_T}\right) = \sigma^2 T\]

Then,

\[Q(S_T) = \frac{\log{S_T} - \log{S_0} - \left(\mu - \frac{1}{2}\sigma^2\right)T}{\sigma \sqrt{T}}\sim \mathrm{Norm}(0,1)\]

Then,

\begin{align*}
P(S_T \geq K) &= P(\log{S_T} \geq \log{K}) \\
&= P\left(\frac{\log{S_T} - \log{S_0} - \left(\mu - \frac{1}{2}\sigma^2\right)T}{\sigma \sqrt{T}} \geq \frac{\log{K} - \log{S_0} - \left(\mu - \frac{1}{2}\sigma^2\right)T}{\sigma \sqrt{T}} \right) \\
&= 1 - P\left(\frac{\log{S_T} - \log{S_0} - \left(\mu - \frac{1}{2}\sigma^2\right)T}{\sigma \sqrt{T}} < \frac{\log{K} - \log{S_0} - \left(\mu - \frac{1}{2}\sigma^2\right)T}{\sigma \sqrt{T}} \right) \\
&= 1 - \mathrm{N}(Q(K)) \\
&= \mathrm{N}(-Q(K))
\end{align*}

where $N$ is the normal distribution function. Then the value $B_0$ is

\[B_0 = \epower{-rT} \mathrm{N}\left(\frac{\log{S_0} - \log{K} + \left(r - \frac{1}{2}\sigma^2\right)T}{\sigma \sqrt{T}}\right) = \epower{-rT} \mathrm{N}(d_2)\]

where $d_2$ is as defined in Hull / slides. We know from risk neutral pricing that $r$ is the risk free rate.

\section{}

The payoffs from this option is perfectly replicated by a portfolio of:

\begin{itemize}
\item long 1 vanilla call option with strike $K$
\item long $K$ digital call option with strike $K$
\end{itemize}

This is shown by the payoffs: let $A$ be the binary asset-or-nothing call.

\begin{align*}
A &= \epower{-rT}\expectation{S\theta(S-K)} \\
&= \epower{-rT}\expectation{(S-K)\theta(S-K) + K\theta(S-K)} \\
&= \epower{-rT}\expectation{\max{(S-K, 0)}} + K\epower{-rT}\expectation{\theta(S-K)} \\
&= C + KB \\
&= S_0 \mathrm{N}(d_1) - K\epower{-rT}\mathrm{N}(d_2) + K\epower{-rT}\mathrm{N}(d_2) \\
&= S_0 \mathrm{N}(d_1)
\end{align*}

where $d_1$ is as defined in Hull / slides.

\section{}

The European Call value using BSM is calculated as:

\begin{lstlisting}
d1 = function(S, K, r, sigma, T) {
  return((log(S/K) + (r+(sigma^2)/2)*T)/(sigma*sqrt(T)))
}

d2 = function(S, K, r, sigma, T) {
  return(d1(S, K, r, sigma, T) - sigma*sqrt(T))
}

bsm_call = function(sigma, K, S, r, T) {
  d1_val = d1(S, K, r, sigma, T)
  d2_val = d2(S, K, r, sigma, T)
  return(list(price=S*pnorm(d1_val) - K*exp(-r*T)*pnorm(d2_val), delta=pnorm(d1_val)))
}
\end{lstlisting}

\begin{lstlisting}
> bsm_call(sigma=0.1, T=1, r=0.08, K=51, S=50*exp(-0.1))
$price
[1] 1.066008

$delta
[1] 0.3639102
\end{lstlisting}

For an American option, let the dividend yield be $d$ and the risk free probability be $q$.

\[qS_0u + (1-q)S_0d = S_0\epower{(r-d)\Delta t}\]

Setting $q$ to be the subject,

\[q = \frac{\epower{(r-d)\Delta t} - d}{u-d}\]

In other words, all we have to do to account for the continuous dividend yield is to set $q$ to the new quantity.

Then, we modify the binomial program we used in the previous homework. The stock tree building procedure remains the same:

\begin{lstlisting}
build_stock_tree = function(S, sigma, T, N) {
  delta_t = T/N
  tree = matrix(0, nrow=N+1, ncol=N+1)
  
  u = exp(sigma*sqrt(delta_t))
  d = exp(-sigma*sqrt(delta_t))
  
  for (i in 1:(N+1)) {
    for (j in 1:i) {
      tree[i,j] = S * u^(j-1) * d^((i-1)-(j-1))
    }
  }
  return(tree)
}
\end{lstlisting}

Particularly, $q$ calculation now accounts for dividends.

\begin{lstlisting}
q_prob = function(r, delta_t, sigma, div=0) {
  u = exp(sigma*sqrt(delta_t))
  d = exp(-sigma*sqrt(delta_t))
  
  return((exp((r-div)*delta_t) - d)/(u-d))
}
\end{lstlisting}

As for building the option tree, I chose to use 4 trees (matrices) instead of a proper data structures (which I'd have used in any other language with proper data structures and object orientedness like \texttt{Python} or \texttt{Java} or even \texttt{JavaScript}). Unfortunately, I am using \texttt{R} and am too lazy to switch languages.

\begin{lstlisting}
  delta_t = T / N
  q = q_prob(r, delta_t, sigma, div=div)
  
  underlying_tree = tree
  value_ne_tree = matrix(0, nrow=nrow(tree), ncol=ncol(tree))
  value_e_tree = matrix(0, nrow=nrow(tree), ncol=ncol(tree))
  option_value_tree = matrix(0, nrow=nrow(tree), ncol=ncol(tree))
  
  ## ONLY CALL OPTIONS NOW
  
  # set not exercise value to 0
  value_ne_tree[nrow(tree),] = 0
  # set exercise value to S - X
  value_e_tree[nrow(tree),] = tree[nrow(tree), ] - X
  # set option value to whichever is higher
  option_value_tree[nrow(tree),] = pmax(value_ne_tree[nrow(tree),], value_e_tree[nrow(tree),])
  
  
  for (i in (nrow(tree)-1):1) {
    for(j in 1:i) {
      # not exercise value is probability weighted discounted
      value_ne_tree[i, j] = ((1-q)*option_value_tree[i+1,j] + q*option_value_tree[i+1,j+1])/exp(r*delta_t)
      # exercise value = S - X
      value_e_tree[i, j] = tree[i, j] - X
      # set option value to whichever is higher
      option_value_tree[i, j] = max(value_ne_tree[i, j], value_e_tree[i, j])
    }
  }
  
  delta = (option_value_tree[2, 2] - option_value_tree[2, 1])/(tree[2, 2] - tree[2, 1])
    
  return(list(tree=tree, value_ne_tree=value_ne_tree, value_e_tree=value_e_tree, option_value_tree=option_value_tree, q=q, delta=delta))
}
\end{lstlisting}

Running the code using the following:

\begin{lstlisting}
sigma=0.1
S=50
T=1
N=100
X=51
r=0.08
div=0.1

stock_tree = build_stock_tree(S=S, sigma=sigma, T=T, N=N)
option_tree = value_binomial_option(stock_tree, sigma=sigma, N=N, T=T, r=r, X=X, type="call", div=div)
\end{lstlisting}

This gave the following results (compared with European BSM pricing):

\begin{itemize}
\item European Call using BSM: $P = 1.066008$, $\Delta = 0.3639102$
\item American Call using Binomial Tree: $P = 1.157862$, $\Delta = 0.3698668$
\end{itemize}

The results are in the following pages:

\begin{landscape}

\subsection{Stock Tree}

\begin{lstlisting}
> round(option_tree$tree[1:11, 1:11], 2)
       [,1]  [,2]  [,3]  [,4]  [,5]  [,6]  [,7]  [,8]  [,9] [,10] [,11]
 [1,] 50.00  0.00  0.00  0.00  0.00  0.00  0.00  0.00  0.00  0.00  0.00
 [2,] 49.50 50.50  0.00  0.00  0.00  0.00  0.00  0.00  0.00  0.00  0.00
 [3,] 49.01 50.00 51.01  0.00  0.00  0.00  0.00  0.00  0.00  0.00  0.00
 [4,] 48.52 49.50 50.50 51.52  0.00  0.00  0.00  0.00  0.00  0.00  0.00
 [5,] 48.04 49.01 50.00 51.01 52.04  0.00  0.00  0.00  0.00  0.00  0.00
 [6,] 47.56 48.52 49.50 50.50 51.52 52.56  0.00  0.00  0.00  0.00  0.00
 [7,] 47.09 48.04 49.01 50.00 51.01 52.04 53.09  0.00  0.00  0.00  0.00
 [8,] 46.62 47.56 48.52 49.50 50.50 51.52 52.56 53.63  0.00  0.00  0.00
 [9,] 46.16 47.09 48.04 49.01 50.00 51.01 52.04 53.09 54.16  0.00  0.00
[10,] 45.70 46.62 47.56 48.52 49.50 50.50 51.52 52.56 53.63 54.71  0.00
[11,] 45.24 46.16 47.09 48.04 49.01 50.00 51.01 52.04 53.09 54.16 55.26
\end{lstlisting}

\subsection{Not Exercise Value}

\begin{lstlisting}
> round(option_tree$value_ne_tree[1:11, 1:11], 2)
      [,1] [,2] [,3] [,4] [,5] [,6] [,7] [,8] [,9] [,10] [,11]
 [1,] 1.16 0.00 0.00 0.00 0.00 0.00 0.00 0.00 0.00  0.00  0.00
 [2,] 0.98 1.35 0.00 0.00 0.00 0.00 0.00 0.00 0.00  0.00  0.00
 [3,] 0.82 1.15 1.56 0.00 0.00 0.00 0.00 0.00 0.00  0.00  0.00
 [4,] 0.68 0.97 1.34 1.80 0.00 0.00 0.00 0.00 0.00  0.00  0.00
 [5,] 0.56 0.81 1.13 1.55 2.07 0.00 0.00 0.00 0.00  0.00  0.00
 [6,] 0.46 0.67 0.96 1.32 1.79 2.37 0.00 0.00 0.00  0.00  0.00
 [7,] 0.37 0.55 0.80 1.12 1.54 2.06 2.70 0.00 0.00  0.00  0.00
 [8,] 0.30 0.45 0.66 0.94 1.31 1.78 2.36 3.05 0.00  0.00  0.00
 [9,] 0.23 0.36 0.54 0.79 1.11 1.53 2.05 2.69 3.45  0.00  0.00
[10,] 0.18 0.29 0.44 0.65 0.93 1.30 1.77 2.35 3.05  3.87  0.00
[11,] 0.14 0.23 0.35 0.53 0.78 1.10 1.52 2.04 2.68  3.44  4.34
\end{lstlisting}

\newpage

\subsection{Exercise Values}

\begin{lstlisting}
> round(option_tree$value_e_tree[1:11, 1:11], 2)
       [,1]  [,2]  [,3]  [,4]  [,5]  [,6] [,7] [,8] [,9] [,10] [,11]
 [1,] -1.00  0.00  0.00  0.00  0.00  0.00 0.00 0.00 0.00  0.00  0.00
 [2,] -1.50 -0.50  0.00  0.00  0.00  0.00 0.00 0.00 0.00  0.00  0.00
 [3,] -1.99 -1.00  0.01  0.00  0.00  0.00 0.00 0.00 0.00  0.00  0.00
 [4,] -2.48 -1.50 -0.50  0.52  0.00  0.00 0.00 0.00 0.00  0.00  0.00
 [5,] -2.96 -1.99 -1.00  0.01  1.04  0.00 0.00 0.00 0.00  0.00  0.00
 [6,] -3.44 -2.48 -1.50 -0.50  0.52  1.56 0.00 0.00 0.00  0.00  0.00
 [7,] -3.91 -2.96 -1.99 -1.00  0.01  1.04 2.09 0.00 0.00  0.00  0.00
 [8,] -4.38 -3.44 -2.48 -1.50 -0.50  0.52 1.56 2.63 0.00  0.00  0.00
 [9,] -4.84 -3.91 -2.96 -1.99 -1.00  0.01 1.04 2.09 3.16  0.00  0.00
[10,] -5.30 -4.38 -3.44 -2.48 -1.50 -0.50 0.52 1.56 2.63  3.71  0.00
[11,] -5.76 -4.84 -3.91 -2.96 -1.99 -1.00 0.01 1.04 2.09  3.16  4.26
\end{lstlisting}

\subsection{Option Value}

Option value is the max of exercise value or not exercise value at each stage.

\begin{lstlisting}
> round(option_tree$option_value[1:11, 1:11], 2)
      [,1] [,2] [,3] [,4] [,5] [,6] [,7] [,8] [,9] [,10] [,11]
 [1,] 1.16 0.00 0.00 0.00 0.00 0.00 0.00 0.00 0.00  0.00  0.00
 [2,] 0.98 1.35 0.00 0.00 0.00 0.00 0.00 0.00 0.00  0.00  0.00
 [3,] 0.82 1.15 1.56 0.00 0.00 0.00 0.00 0.00 0.00  0.00  0.00
 [4,] 0.68 0.97 1.34 1.80 0.00 0.00 0.00 0.00 0.00  0.00  0.00
 [5,] 0.56 0.81 1.13 1.55 2.07 0.00 0.00 0.00 0.00  0.00  0.00
 [6,] 0.46 0.67 0.96 1.32 1.79 2.37 0.00 0.00 0.00  0.00  0.00
 [7,] 0.37 0.55 0.80 1.12 1.54 2.06 2.70 0.00 0.00  0.00  0.00
 [8,] 0.30 0.45 0.66 0.94 1.31 1.78 2.36 3.05 0.00  0.00  0.00
 [9,] 0.23 0.36 0.54 0.79 1.11 1.53 2.05 2.69 3.45  0.00  0.00
[10,] 0.18 0.29 0.44 0.65 0.93 1.30 1.77 2.35 3.05  3.87  0.00
[11,] 0.14 0.23 0.35 0.53 0.78 1.10 1.52 2.04 2.68  3.44  4.34
\end{lstlisting}

\end{landscape}

\section{}

\subsection{}

The only part of the program we need to modify is:

\begin{lstlisting}
value_binomial_option = function(tree, sigma, N, T, r, alpha=3, div=0) {
  delta_t = T / N
  q = q_prob(r, delta_t, sigma, div=div)
  
  option_tree = matrix(0, nrow=nrow(tree), ncol=ncol(tree))
  
  # option pays S^alpha at expiration
  option_tree[nrow(tree),] = tree[nrow(tree), ]^alpha
  
  for (i in (nrow(tree)-1):1) {
    for(j in 1:i) {
      option_tree[i, j] = ((1-q)*option_tree[i+1,j] + q*option_tree[i+1,j+1])/exp(r*delta_t)
    }
  }
  
  delta = (option_tree[2, 2] - option_tree[2, 1])/(tree[2, 2] - tree[2, 1])
  
  return(list(tree=tree, option_tree=option_tree, price=option_tree[1,1], q=q, delta=delta))
}
\end{lstlisting}

to account for the introduction of $\alpha$ and the different expiration price of power options.

Running this:

\begin{lstlisting}
sigma=0.3
S=100
T=1/4
N=100
r=0.1
stock_tree = build_stock_tree(S=S, sigma=sigma, T=T, N=N)
option_tree = value_binomial_option(stock_tree, sigma=sigma, N=N, T=T, r=r, alpha=3)
\end{lstlisting}

We find that the price of the option is $P=1124610.5$. We also obtain the following trees:

\begin{landscape}

\subsubsection{Stock Tree}

\begin{lstlisting}
> round(option_tree$tree[1:11, 1:11], 2)
        [,1]   [,2]   [,3]   [,4]   [,5]   [,6]   [,7]   [,8]   [,9]  [,10]  [,11]
 [1,] 100.00   0.00   0.00   0.00   0.00   0.00   0.00   0.00   0.00   0.00   0.00
 [2,]  98.51 101.51   0.00   0.00   0.00   0.00   0.00   0.00   0.00   0.00   0.00
 [3,]  97.04 100.00 103.05   0.00   0.00   0.00   0.00   0.00   0.00   0.00   0.00
 [4,]  95.60  98.51 101.51 104.60   0.00   0.00   0.00   0.00   0.00   0.00   0.00
 [5,]  94.18  97.04 100.00 103.05 106.18   0.00   0.00   0.00   0.00   0.00   0.00
 [6,]  92.77  95.60  98.51 101.51 104.60 107.79   0.00   0.00   0.00   0.00   0.00
 [7,]  91.39  94.18  97.04 100.00 103.05 106.18 109.42   0.00   0.00   0.00   0.00
 [8,]  90.03  92.77  95.60  98.51 101.51 104.60 107.79 111.07   0.00   0.00   0.00
 [9,]  88.69  91.39  94.18  97.04 100.00 103.05 106.18 109.42 112.75   0.00   0.00
[10,]  87.37  90.03  92.77  95.60  98.51 101.51 104.60 107.79 111.07 114.45   0.00
[11,]  86.07  88.69  91.39  94.18  97.04 100.00 103.05 106.18 109.42 112.75 116.18
\end{lstlisting}

\subsubsection{Option Tree}

\begin{lstlisting}
> round(option_tree$option_tree[1:11, 1:11], 2)
           [,1]      [,2]      [,3]      [,4]    [,5]    [,6]    [,7]    [,8]    [,9]   [,10]   [,11]
 [1,] 1124610.5       0.0       0.0       0.0       0       0       0       0       0       0       0
 [2,] 1073862.9 1174993.2       0.0       0.0       0       0       0       0       0       0       0
 [3,] 1025405.3 1121972.1 1227633.1       0.0       0       0       0       0       0       0       0
 [4,]  979134.4 1071343.7 1172236.7 1282631.2       0       0       0       0       0       0       0
 [5,]  934951.4 1022999.8 1119340.0 1224753.1 1340093       0       0       0       0       0       0
 [6,]  892762.1  976837.3 1068830.3 1169486.6 1279622 1400130       0       0       0       0       0
 [7,]  852476.6  932758.0 1020599.8 1116714.1 1221880 1336949 1462856       0       0       0       0
 [8,]  814009.0  890667.7  974545.7 1066322.9 1166743 1276620 1396845 1528392       0       0       0
 [9,]  777277.2  850476.7  930569.8 1018205.5 1114094 1219013 1333813 1459424 1596864       0       0
[10,]  742202.9  812099.4  888578.2  972259.4 1063821 1164006 1273625 1393568 1524806 1668404       0
[11,]  708711.3  775453.7  848481.5  928386.7 1015817 1111481 1216154 1330684 1456000 1593118 1743149
\end{lstlisting}

\end{landscape}

\subsection{}

Let $P$ be the power option.

\[P = S^\alpha\]

$dS = \mu S dt + \sigma S dZ$. Recall Ito's lemma which states that

\[dP = \left(\defrac{P}{S}\mu S + \defrac{P}{t} + \frac{1}{2}\dedefrac{P}{S} \sigma^2 S^2 \right)dt + \defrac{P}{S} \sigma S dZ\]

Now,

\begin{align*}
\defrac{P}{S} &= \alpha S^{\alpha - 1} \\
\dedefrac{P}{S} &= \alpha(\alpha - 1)S^{\alpha - 2} \\
\defrac{P}{t} &= 0
\end{align*}

Then,

\begin{align*}
dP &= \left(\mu\alpha S^\alpha + \frac{\alpha(\alpha - 1)}{2}\sigma^2 S^\alpha \right)dt + \alpha\sigma S^\alpha dZ \\
&= \left(\mu\alpha P + \frac{\alpha(\alpha - 1)}{2}\sigma^2 P \right)dt + \alpha\sigma P dZ
\end{align*}

\[\frac{dP}{P} = \left(\mu\alpha + \frac{\alpha(\alpha - 1)}{2}\sigma^2 \right)dt + \alpha\sigma dZ\]

Then we know that $\frac{dP}{P}$ is a Wiener process with drift constant $\left(\mu\alpha + \frac{\alpha(\alpha - 1)}{2}\sigma^2 \right)$ and variance $(\alpha\sigma)^2$.

Then, we know that the expected change between time $0$ and time $T$ is

\[\mathrm{E}(P_T) = P_0 \exp{\left[\left(\mu\alpha + \frac{\alpha(\alpha - 1)}{2}\sigma^2 \right)T\right]}\]

Let the value of the option be $V$.

\[V = \epower{-rT} \expectation{P_T} = \epower{-rT}P_0 \exp{\left[\left(\mu\alpha + \frac{\alpha(\alpha - 1)}{2}\sigma^2 \right)T\right]}\]

\begin{lstlisting}
black_scholes_power = function(alpha, r, T, sigma, S) {
  discount = exp(-r*T)
  p0 = S^alpha
  expectation = exp((r*alpha + (alpha * (alpha - 1)) / 2 * sigma^2)*T)
  return(discount*p0*expectation)
}

black_scholes_power(alpha=3, r=0.1, T=0.25, sigma=0.3, S=100)
\end{lstlisting}

Now $P_0 = 100^3 = 1000000$, $r = 0.1$, $\mu = r = 0.1$ by risk neutral pricing, $T = 0.25$, $\alpha = 3$, $\sigma = 0.3$.

Then, $V = 1124682$



\end{document}